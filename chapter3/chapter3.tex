\chapter{NLP2SQL}
\label{Ch-3:Sec:Extraction}

In this chapter, we begin to set up the NLP module required in the Q\&A system. 

\section{Problem Statement}

Now we have a cleaned dataset which includes vertices $V$ and their attributes $A$. Giving a better way to connect with the MySQL database of the wiki farm. Thus, we decide to change the natural language question to a SQL query. The relationships between $V$ and $A$ are changed into a relational database $T$. each attribute has its own table $t \in T$.\\
The graph with labels and attributes is still kept for the basic searching, giving the user an intuitive grasp of what they want. To enhance the comprehension of search engine, we annotate manually 100 questions with its corresponding SQL queries, like the WikiSQL from \cite{yin2015neural}.

\section{Related Work}
(unfinished)

Neural Enquirer: Learning to Query Tables in Natural Language
\cite{yin2015neural}

Wikipedia-based Semantic Interpretation for Natural Language Processing
\cite{gabrilovich2009wikipedia}

fr2sql: Interrogation de bases de données en français
\cite{couderc2015fr2sql}

\section{Summary of Technique}

\subsection{SQLite with sqlite3}

SQLite\footnote{https://www.sqlite.org/index.html} can build lightweight disk-based SQL databases not requiring installing separate server process. However, SQLite is a C library which needs interface to interact with other programming language. Thus, we choose sqlite3\footnote{https://docs.python.org/3.4/library/sqlite3.html} to finish this part of work. The sqlite3 module is a DB-API 2.0 interface for SQLite databases written by Gerhard Häring supporting Python language.

\subsection{Seq2SQL}

Seq2SQL is a deep neural network introduced in the paper \cite{yin2015neural}. In this paper, the author gives a model of translating natural language questions to corresponding SQL queries. It is trained by the WikiSQL, a dataset of hand-annotated examples of questions with their corresponding SQL, which gives rewards to learn a better policy to generate the query. Also, Seq2SQL can offer an advantage of the SQL structure and simplify the original problem.

\section{Conclusion}

(unfinished)